%                                                                                       
% This document is available under the Creative Commons Attribution-ShareAlike
% License; additional terms may apply. See
%   * http://creativecommons.org/licenses/by-sa/3.0/
%   * http://creativecommons.org/licenses/by-sa/3.0/legalcode
%
% Created: 2011-08-14 17:43:38+02:00
% Main authors:
%     - Jérôme Pouiller <jezz@sysmic.org>
%

\part{Problèmatique}

%\begin{frame}
%  \partpage
%\end{frame}

%\begin{frame}
%  \tableofcontents[currentpart]
%\end{frame}

\begin{frame}{Qu'est-ce que le temps réel ?}
  Deux exemples :
  \begin{itemize} 
  \item Un  système de navigation calculant le  meilleur parcours d'un
    navire.
  \item Un système de navigation  calculant la position d'un navire en
    mouvement.
  \end{itemize}     
  \begin{itemize}
  \item  Dans le  premier cas  le  temps nécessaire  à l'obtention  du
    calcul est accessoire.
  \item Dans le deuxième cas, si  le calcul est trop long, la position
    est erronée.
  \end{itemize}
\end{frame}

\begin{frame}{Le système temps réel et son environnement}
  \textbf{Système}  :  ensemble d'"activités"  correspondant  à un  ou
  plusieurs traitements effectués en séquence ou en concurrence et qui
  communiquent  éventuellement entre eux.\\[3mm]

  Un système temps réel interagit avec son environnement
  \begin{itemize}
  \item Capteurs : signaux et mesure de signaux
  \item Unité de traitement
  \item Actionneurs : actions sur l'environnement à des moment précis
  \end{itemize}
\end{frame}

\begin{frame}{Une définition informelle}
  \textbf{Système  temps réel  }  : un  système  dont le  comportement
  dépend,  non seulement  de l'exactitude  des  traitements effectués,
  mais également
  du temps où les résultats de ces traitements sont produits.\\[3mm]

  En d'autres  termes, un  retard est considéré  comme une  erreur qui
  peut entrainer de graves  conséquences.  

  \note{Un système temps réel n'est  pas forcement rapide (il est meme
    souvent plutot sous optimal dans le cas moyen)}
\end{frame}

\begin{frame}{Echéances et systèmes temps réel}
  On distingue différents types d'échéances
  \begin{itemize}
  \item  \textbf{Echéance dure}  :  un retard  provoque une  exception
    (traitement d'erreur) (exemple:  carte son)
  \item \textbf{Echéance molle  ou lâche} : un retard  ne provoque pas
    d'exception (exemple: IHM)
  \end{itemize}

  On distingue par conséquent différents types de systèmes temps réels
  \begin{itemize}
  \item \textbf{Temps  réel dur} :  les échéances ne doivent  en aucun
    cas  être  dépassées 
  \item \textbf{Temps réel lâche  ou mou} : le dépassement occasionnel
    des échéances ne met pas  le système en péril
  \end{itemize}
\end{frame}

\begin{frame}{Criticité}
  En plus de la tolérance à l'erreur, nous devons prendre en compte la
  criticité de l'erreur:
  \begin{itemize} 
  \item Téléphone portable
  \item Carte son professionnelle
  \item Système de commande d'un robot industriel
  \item Système de commande d'un avion de ligne
  \end{itemize} 
\end{frame}

% \begin{frame}{Approche centralisée}
%   Un processeur traite l'inforation issue de capteurs et envoie le résultat vers
%   les actionneurs\\[1cm]
%   
%   \begin{center}
%     \includegraphics[width = 12cm]{centralise.eps}
%   \end{center}
% \end{frame}
% 
% \begin{frame}{Approche multi-systèmes}
%   Une système contrôlé et un système contrôleur interagissant avec des capteurs et
%   des contrôleurs\\[3mm]
%   
%   \begin{center}
%     \includegraphics[width = 8cm]{multisysteme.eps}
%   \end{center}
% \end{frame}
% 
% 
% \begin{frame}{Cadencement temporel}
%   \begin{itemize}
%   \item selon une mesure du temps (système piloté par le temps -
%     \textsl{time-driven system})
%   \item selon des événements (système piloté par les événements -
%     \textsl{event-driven system})
%   \item réponse en temps limité : système réactif
%   \end{itemize}
% \end{frame}


\begin{frame}{Prévisibilité, déterminisme, fiabilité}

  Systèmes temps réels : prévus pour le pire cas.\\[3mm]

  \textbf{Prévisibilité} : Pouvoir déterminer à l'avance si un système
  va pouvoir  respecter ses contraintes  temporelles. Connaissance des
  paramètres liés  aux calculs.\\[3mm]

  \textbf{Déterminisme}   :   Enlever   toute   incertitude   sur   le
  comportement  des  tâches  individuelles  et sur  leur  comportement
  lorsqu'elles sont mises ensemble.
  \begin{itemize}
  \item variation des durées d'exécution des tâches
  \item durée des E/S, temps de réaction
  \item réaction aux interruptions, etc.
  \end{itemize}

  \textbf{Fiabilité} : comportement et tolérance aux fautes.
\end{frame} 

\begin{frame}{Autour du temps réel}
  Par conséquent, le temps réel possède plusieurs facettes:
  \begin{itemize} 
  \item Ingénierie informatique: Algorithmique, développement
  \item  Ingérierie   électromécanique:  Maitrise  de  l'environnement
    physique
  \item Processus: Maitrise de la qualité du produit, garantie sur les
    bugs
  \item Administrative: Certification
  \end{itemize} 
  \textbf{Exemple} : Un airbag est un système temps réel très dur avec
  une échéance de temps très faible. La puissance du système n'est pas
  dans sa  conception, mais dans  sa garantie de qualité.   C'est très
  facile  de faire  un  airbag,  c'est beaucoup  plus  complexe de  le
  garantir.
\end{frame}


\begin{frame}{Exemple - Combiné GSM}
  Le système doit combiner :
  \begin{itemize}
  \item Couche physique : émission, réception, activation du vocodeur,
    mesures du niveau de réception, etc.
  \item  Procédures  logicielles  :  communication avec  les  bases  de
    données  du  système  pour  les  mises  à  jour  de  localisation,
    transmission  des résultats  de mesure  de qualité,  scrutation de
    messages d'appel, etc.
  \end{itemize}

  Les  contraintes de temps  : 577  $\mu$s de  parole émis  puis recus
  toutes les  4,6 ms.\\[3mm]

  Les  contraintes de  mobilité  :  emettre plus  tôt  si la  distance
  augmente,  plus tard si elle diminue.\\[3mm]

  Le  système  doit  être  assez  réactif pour  que  l'utilisateur  ne
  s'appercoive de rien (~100ms).

\end{frame}


% \begin{frame}{Limite des systèmes classiques}
%   Les systèmes classiques s'appuient sur un système d'exploitation en général
%   mal adaptés au temps réel.
%   \begin{itemize}
%   \item Politique d'ordonnancement visant à équilibrer équitablement
%     le temps alloué à chaque tâche
%   \item mécanismes d'accès aux ressources partagées et de
%     synchronisation comportent des incertitudes temporelles
%   \item gestion des interruptions non optimisées
%   \item la gestion de la mémoire virtuelle, des caches engendrent
%     des fluctuations temporelles
%   \item la gestion des temporisateurs qui servent à la manipulation
%     du temps pas assez fin
%   \end{itemize}
% \end{frame}


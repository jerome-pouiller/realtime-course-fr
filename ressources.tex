% Protection des ressources, structure de donnée, ressource
% matériel, mutex, dead lock, latente introduite, inversion de
% priorité, priority ceilling, priority inhéritence,
% semaphore, condition, queue, barrier, RCU,
\part{Partage de ressources}

\section{Ordonnancement avec contraintes de précédance}

\begin{frame}{Contraintes de précédence (1)} 
  \begin{itemize}
  \item ici, seulement précédence simple: si Ti est périodique de période Pi, alors Tj l'est aussi et Pj 
  \item principe de l'établissement de l'ordonnancement :
    \begin{itemize} 
    \item transformer l'ensemble des tâches dépendantes en un ensemble de tâches \textbf{indépendantes} que l'on ordonnancera par un algorithme classique
    \item par des modifications des paramètres des tâches : si $T_i \leftarrow T_j$ alors la règle de transformation doit respecter:
      \begin{itemize} 
      \item $r_j \le r_i$
      \item $Prio_i < Prio_j$
      \end{itemize} 
    \item validation de l'ordonnançabilité selon des critères utilisés pour des tâches indépendantes
    \end{itemize}
  \end{itemize}
\end{frame}

\begin{frame}{Contraintes de précédence (2)} 
  Contraintes de précédence et Rate Monotonic:
  la transformation s'opère hors ligne sur la date de réveil et sur les délais critiques
  \begin{itemize} 
  \item $r^*_i = Max{r_i, r^*_j}$ pour tous les $j$ tels que $Tj →  Ti$
  \item si $T_i  → T_j$ alors $Prio_i > Prio_j$ dans  le respect de la
    règle d'affectation des priorités par Rate Monotonic
  \end{itemize}
\end{frame} 

\begin{frame}{Contraintes de précédence (2)} 
  Contraintes de  précédence et Deadline  Monotonic: la transformation
  s'opère hors ligne sur la date de réveil et sur les délais critiques
  \begin{itemize}
  \item $r^*_i = Max{r_i, r^*_j}$ pour tous les j tels que $Tj \leftarrow Ti$
  \item $D*i = Max{Di, D*j}$ pour tous les j tels que $T_j \leftarrow  T_i$
  \item  si $T_i  \leftarrow  T_j$  alors $Prio_i  >  Prio_j$ dans  le
    respect de la règle d'affectation des priorités par DMA
  \end{itemize} 
\end{frame} 

\begin{frame}{Contraintes de précédence (2)} 
  \begin{itemize}
  \item Contraintes de précédence 
    \begin{itemize}
    \item sur l'ordre d'exécution des  tâches les unes par rapport aux
      autres
    \item généralement décrites par un graphe orienté G 
      \begin{itemize}
      \item $Ja < Jb$ indique que la tâche Ja est un prédécesseur de Jb 
      \item $Ja → Jb$ indique que la tâche Ja est un prédécesseur immédiat de Jb 
      \end{itemize}
    \end{itemize}
  \end{itemize}
\end{frame}

\begin{frame}{Contraintes de précédence (3)} 
  contraintes de précédence et EDF
  \begin{itemize}
  \item modification des échéances de façon à ce qu'une tâche ait toujours un di inférieur à celui de ses successeurs (algorithme de Chetto et al.)
  \item une tâche ne doit être activable que si tous ses prédécesseurs ont terminé leur exécution 
  \item modification de la date de réveil et de l'échéance
    \begin{itemize}
    \item  r*i = Max{ri, Max{r*j + Cj}} pour tous les j tels que Tj → Ti
    \item d*i = Min{di, Min{d*j - Cj}} pour tous les j tels que Ti → Tj
    \item on itère sur les prédécesseurs et successeurs immédiats
    \item on commence les calculs par les tâches qui n'ont pas de prédécesseurs pour le calcul des r et par les tâches qui n'ont pas de successeur pour le calcul des d
    \end{itemize}
  \end{itemize}
\end{frame}

\section{Problèmatique des accès concurents}

\subsection{Protection des structures de données}

\begin{frame}[fragile]{Exemple de partage de données}
  \begin{columns}
    \begin{column}{5cm}
      Soient \c{f1} et \c{f2}:
      \begin{lstlisting}
int a = 1;
void f1() {
  a++;
}
void f2() {
  a++;
}
       \end{lstlisting}
     \end{column}
     \begin{column}{5cm}
       \c{f1} en assembleur:
       \begin{lstlisting}
mov r1, $123456
inc r1
mov $123456, r1
       \end{lstlisting}
       \c{f2} en assembleur:
       \begin{lstlisting} 
mov r2, $123456
inc r2
mov $123456, r2
      \end{lstlisting}
    \end{column}
  \end{columns}
\end{frame} 

\begin{frame}[fragile]{Exemple de partage de données}
  Prennons le  cas où  les fonctions \c{f1}  et \c{f2}  sont éxecutées
  dans deux tâches différentes.  \c{f1} commence par s'éxecuter:
\begin{lstlisting} 
mov r1, $123456
inc r1
  \end{lstlisting} 
  A  ce  moment \c{r1}  vaut  donc 2,  alors  que  la cellule  mémoire
  \c{\$123456}  vaut encore 1.   L'ordonnanceur préempte  alors \c{f1}
  afin de donner la main à \c{f2}.  L'ordonnanceur sauve les registres
  sur la pile et change le contexte:
\begin{lstlisting} 
mov r1, $123456
inc r2
mov $123456, r2
[...]
  \end{lstlisting} 
\end{frame}

\begin{frame}[fragile]{Exemple de partage de données}
  \c{\$123456}  contient  maintenant 2.   La  tâche  2  se termine  et
  l'ordonanceur rend la main à la tâche 1:
  \begin{lstlisting} 
mov $123456, r1
  \end{lstlisting} 
  En  toute  logique,  après  l'éxécution  de  \c{f1}  et  de  \c{f2},
  \c{\$123456} devrait contenir 3. Or il ne contient que 2.
\end{frame} 

\subsection{Protection des ressources matériel}

\begin{frame}[fragile]{Exemple de ressource partagée}
  Cas d'un  périphérique réseau avec des registres  mappés en mémoire.
  Le  registre  \c{0xABC0}  permet  de  placer la  donnée  à  envoyer.
  L'écriture  d'un 1  sur  le registre  \c{0xABC4} permet  d'effectuer
  l'envoi:
\begin{lstlisting} 
void send(int data) {
  *0xABC0 = data;
  *0xABC4 = 1;
}
  \end{lstlisting} 
\end{frame} 

\begin{frame}[fragile]{Exemple de ressource partagée}
  Etudions le cas de l'éxecution simultanée de cette fonction par deux
  tâches.  La première tâche appelle \c{send} avec \cmd{data = 42}:
  \begin{lstlisting} 
*0xABC0 = 42
  \end{lstlisting} 
  La  tâche est  préemptée.  La  seconde tâche  appelle  \c{send} avec
  \cmd{data = 10}:
\begin{lstlisting} 
*0xABC0 = 10
*0xABC4 = 1
\end{lstlisting} 
  \c{10} est envoyé. La tâche 1 reprend la main:
\begin{lstlisting} 
*0xABC4 = 1
\end{lstlisting} 
  \c{10} (au lieu de \c{42}) est de nouveau envoyé.\\[3mm]

  Ces cas sont aussi valable dans le cas d'une interruption.
\end{frame} 

\subsection{Réentrance}

\begin{frame}{Comment éviter le problème?}
  Les problèmes  d'accès concurrents se  traduisent très souvent  par des
  \emph{races  conditions}.  C'est  à  dire des  problèmes  aléatoires
  produit par une séquence particulière d'évènements
  \begin{itemize} 
  \item   Les  \emph{race  conditions}   sont  souvent   difficiles  à
    reproduire et à identifier
  \item Les  \emph{race conditions} peuvent être latente  dans le code
    et se déclarer suite à une modification de l'environnement externe
  \item Une race condition coûte chère (diffuculté de correction, peut
    potentiellement atterrir en production)
  \end{itemize} 
  Comment s'en protèger?
  \begin{itemize} 
  \item Ne pas utiliser de variables globales ou de ressources partagées
  \item Utiliser des accès atomiques
  \item  Placer des  accès aux  ressources partagée  dans  des \emph{sections
    critiques}
  \end{itemize} 
  Une  fonction  pouvant  être  appellée simultanénement  depuis  deux
  contextes de tâches différentes est dite \emph{réentrante}
\end{frame} 

\begin{frame}{Partage de ressources critiques} 
  \begin{itemize}
  \item Une ressource critique ne peut :
    \begin{itemize}
    \item être utilisée simultanément par plusieurs tâches 
    \item être réquisitionnée par une autre tâche 
    \end{itemize}
  \item Notion de section critique :
    \begin{itemize}
    \item  séquence d'instructions pendant  lesquelles on  utilise une
      ressource critique
      \note{hein?}
    \item sans problème dans le cas d'un ordonnancement non préemptif,
      mais c'est  rarement le cas  dans un environnement temps  réel ⇒
      évaluation du temps de réponse très difficile, sinon impossible
      \begin{itemize}
      \item abondante littérature ! 
      \end{itemize}
    \end{itemize}
  \end{itemize}
\end{frame}

\subsection{Partage de ressource entre deux tâches: Fonctionnement d'un mutex}

\begin{frame}[fragile]{Fonctionnement d'un mutex}
  Nécessite une instruction assembleur  premettant un accès en lecture
  et une écriture  en une instruction: \texttt{test\_and\_set} affecte
  le registre d'état  en fonction de la valeur  du registre et affecte
  la valeur  1 au  registre. On peut  délopper la fonction  \c{lock} à
  partir de là:
\begin{lstlisting} 
void lock(mutex_t *m) {
  while (test_and_set(m))
    schedule();
}

void unlock(mutex_t *m) {
  m = 0;
  schedule();
}
\end{lstlisting} 
\end{frame}

\begin{frame}[fragile]{Fonctionnement d'un mutex}
  Un peu mieux:
  \begin{lstlisting} 
void lock(mutex_t *m) {
  while (test_and_set(m)) {
    this_task.reason = m;
    this_task.state = stop;
    schedule();
  }
}

void unlock(mutex_t *m) {
  m = 0;
  foreach (i in tasks)
    if (i.state == stop && i.reason == m)
      i.state = run;
  schedule();
}
  \end{lstlisting}
\end{frame} 

\subsection{Partage de ressource entre interruptions}
\begin{frame}{ressource et interruptions}
\note{Pour l'instant, on passe}
Une interruption s'execute dans le contexte de la tâche courante.
Par conséquent, l'utilisation d'un mutex dans une interruption pourrait entrainer un double lock. 
il ne faut pas utiliser de mutex dans une interruption.
Le partage de ressource avec une interruption nécessite de désactiver temporairement les interruptions.
Ceci entraine de la latence dans les interruptions
\end{frame} 

\subsection{Partage de ressource dans le cadre d'un multicoeur}
\begin{frame}{Dans un multicoeur}
Dans le cas d'un multicoeur, on ne désactive que les interruptions ``locale'' (sur le CPU courant).
Les interruptions peuvent donc s'ecuter sur les autres CPU.
Il faut tout de même protèger la ressource. 


utilisation de Spin lock à la place du mutex ou en plus de la désactivation des interruptions. 

\end{frame} 

\section{Problème liés aux partage de ressources}

\subsection{Dead Lock}

\begin{frame}[fragile]{Dead lock}
  \begin{itemize} 
  \item Aussi appellé \emph{étreinte fatale}
  \item Deux tâches utilisent  deux ressources imbriquées dans l'ordre
    inverses
  \end{itemize} 
  \textbf{Exemple:}
  \begin{columns}
    \begin{column}{6.5cm}
      Tache 1:
      \begin{lstlisting}
lock(m1);
// Preemption par la tache 2 ici
lock(m2);
      \end{lstlisting} 

      Tache 2:
      \begin{lstlisting} 
lock(m2);
// Deadlock ici
lock(m1);
      \end{lstlisting} 
    \end{column}
    \begin{column}{4cm}
      % AABBAXXX
      \begin{tikzpicture}[scale=0.35]
        \timeline{10}{-3.5}{-1.0/A, -2.5/B}
        \fill[color=black!25] (5, -1.5) rectangle +(4, 1);
        \fill[color=black!25] (4, -3.0) rectangle +(5, 1);
        \fill[cgreen] (0,-1.5)  \hi 2 \lo 2 \hi 1 \lo 4;
        \fill[cred]   (0,-3.0)  \lo 2 \hi 2 \lo 1 \lo 4;
        \pattern[pattern=north east lines] (1, -1.5) rectangle +(7, 1);
        \pattern[pattern=north west lines] (3, -3.0) rectangle +(5, 1);
        \draw[cgreen] (0 + 0.5, -1) circle (3pt);
        \draw[cred]   (2 + 0.5, -2.5) circle (3pt);
      \end{tikzpicture}
    \end{column}
  \end{columns}
\end{frame} 

\begin{frame}[fragile]{Dead lock}
  \textbf{Remarque:} \\
  Le code suivant:
  \begin{lstlisting} 
lock();
lock(); 
  \end{lstlisting} 
  entraine un  \emph{double lock},  un type particulier  de \emph{Dead
    lock}
\end{frame} 

\begin{frame}[fragile]{Mutex dans une interruption}
  Ne jamais utiliser de mutex dans une interruption.
  \begin{itemize} 
  \item Si  la ressource  est occuppée par  la tâche qui  vient d'être
    préemptée, le \texttt{lock()} s'éxécutera dans le même contexte
  \item[$\rightarrow$] Double lock
  \item De plus,  le blocage d'un mutex peut  entrainer peut entrainer
    une  très  important latance  ce  qui  est  en contradiction  avec
    l'objectif de rester le minimum de temps dans une interruption
  \item[$\rightarrow$] Règle générale: Il ne faut pas appeller \texttt{schedule} dans une interruption.
  \end{itemize} 
\end{frame} 

\subsection{Latence introduite}

\begin{frame}{Latence}
  \begin{itemize} 
  \item  Cas d'une tâche  de faible  priorité utilisant  une ressource
    d'une tâche de haute priorité.
  \item Existe uniquement en environnement préemptif
  \end{itemize} 
  Exemple avec deux tâches $P(A) < P(B)$:
  \begin{itemize} 
  \item Cas sans partage de ressource:
    % AABBBAAAA
    \begin{center}
    \begin{tikzpicture}[scale=0.35]
        \timeline{10}{-3.5}{-1.0/A, -2.5/B}
         \fill[cgreen] (0,-1.5)  \hi 2 \lo 3 \hi 4 \lo 1;
         \fill[cred]   (0,-3.0)  \lo 2 \hi 3 \lo 4 \lo 1;
         \draw[cgreen] (0 + 0.5, -1) circle (3pt);
         \draw[cred]   (2 + 0.5, -2.5) circle (3pt);
     \end{tikzpicture}
      $$TR_B = 3$$
    \end{center}
  \item Cas avec une ressource partagée:
     % AABAAABBA
    \begin{center}
     \begin{tikzpicture}[scale=0.35]
        \timeline{10}{-3.5}{-1.0/A, -2.5/B}
        \fill[color=black!25] (3, -3.0) rectangle +(3, 1);
        \fill[cgreen] (0,-1.5)  \hi 2 \lo 1 \hi 3 \lo 2 \hi 1 \lo 1;
        \fill[cred]   (0,-3.0)  \lo 2 \hi 1 \lo 3 \hi 2 \lo 1 \lo 1;
        \pattern[pattern=north east lines] (1, -1.5) rectangle +(5, 1);
        \pattern[pattern=north east lines] (6, -3.0) rectangle +(1, 1);
        \draw[cgreen] (0 + 0.5, -1) circle (3pt);
        \draw[cred]   (2 + 0.5, -2.5) circle (3pt);
      \end{tikzpicture}
      $$TR_B = 6$$
    \end{center}
  \end{itemize} 
    
\end{frame} 

\subsection{Inversion de priorité}

\begin{frame}{Inversion de priorité}
  \begin{itemize} 
  \item Phénomène dû à la présence simultanée de priorités fixes et de
    ressources à accès exclusif dans un environnement préemptif
  \item  Une  tâche  de  priorité  intermédiaire peut  être  élue  par
    l'ordonnanceur  alors  que  des  tâches  de  plus  haute  priorité
    attentent.
  \item Il est diffile de calculer  le temps de blocage des tâches par
    une inversion de priorité (il est généralement non-borné)
  \end{itemize} 
  Cas notable de \emph{Path Finder}
\end{frame}

\begin{frame}{Inversion de priorité}
  \textbf{Exemple:}\\
  On ajoute à l'exemple précédant une  tâche $C$ telle que : 
  $P(B) > P(C) > P(A)$
  \begin{itemize} 
  \item   Cas sans partage de ressource:
    % AABBBBCCCCCCCCCAA
    \begin{center}
      \begin{tikzpicture}[scale=0.30]
           \timeline{16}{-5}{-1./A, -2.5/B, -4./C}
           \fill[cgreen] (0,-1.5)  \hi 2 \lo 4 \lo 9 \hi 2;
           \fill[cred]   (0,-3.0)  \lo 2 \hi 4 \lo 9 \lo 2;
           \fill[cblue]  (0,-4.5)  \lo 2 \lo 4 \hi 9 \lo 2;
           \draw[cgreen] (0 + 0.5, -1) circle (3pt);
           \draw[cred]   (2 + 0.5, -2.5) circle (3pt);
           \draw[cblue]  (2 + 0.5, -4) circle (3pt);
         \end{tikzpicture}
    \end{center}
  \item   Cas avec un mutex:
    % AABCCCCCCCCCABBBA
    \begin{center}
      \begin{tikzpicture}[scale=0.30]
        \timeline{16}{-5}{-1./A, -2.5/B, -4./C, }
        \fill[color=black!25] (3, -3.0) rectangle +(10, 1);
        \fill[cgreen] (0,-1.5)  \hi 2 \lo 1 \lo 9 \hi 1 \lo 3 \hi 1;
        \fill[cred]   (0,-3.0)  \lo 2 \hi 1 \lo 9 \lo 1 \hi 3 \lo 1;
        \fill[cblue]  (0,-4.5)  \lo 2 \lo 1 \hi 9 \lo 1 \lo 3 \lo 1;
        \pattern[pattern=north east lines] (1, -1.5) rectangle +(12, 1);
        \pattern[pattern=north east lines] (13, -3.0) rectangle +(1, 1);
        \draw[cgreen] (0 + 0.5, -1) circle (3pt);
        \draw[cred]   (2 + 0.5, -2.5) circle (3pt);
        \draw[cblue]  (2 + 0.5, -4) circle (3pt);
      \end{tikzpicture}
    \end{center}
    La tâche  $C$ s'éxecute alors  qu'elle a une priorité  inféreure à
    $B$ et ne partage aucune ressource avec $B$.
  \end{itemize} 
\end{frame} 

\begin{frame}{Conseils}
  \begin{itemize} 
  \item Essayer de limiter la taille des sections critique
  \item  Essayer  de découper  les  sections  critiques  de manière  à
    favoriser la préemption des tâches de haute priorité
  \item Essayer de  diminuer la granularité des mutex  afin de limiter
    la taille des section critique.
  \end{itemize} 
\end{frame} 

\section{Solutions}

\subsection{Priority inheritence}

\begin{frame}{Héritage de priorité}
  \begin{itemize} 
  \item Aussi appellé \emph{Priority inheritence}
  \item Principe: Si  une tâche bloque une ressource  demandée par une
    tâche  de plus  haute  priorité, elle  acquiert temporairement  la
    priorité de la tâche de haute priorité.
  \item Attention à la gestion de l'héritage en cascade.
  \item  L'algorithme parait  simple,  mais c'est  assez complexe.  Il
    existe même des cas problèmatiques (dont le cout de résolution est
    polynomial)   dans   le   cas   d'architectures   multiprocesseurs
    symétriques sur des \cmd{rw\_lock}.
  \end{itemize} 
  Exemple précédant:
  % AABABBBCCCCCCCCCA
  \begin{center}
    \begin{tikzpicture}[scale=0.30]
      \timeline{16}{-5}{-1./A, -2.5/B, -4./C, }
      \fill[color=black!25] (3, -3.0) rectangle +(1, 1);
      \fill[cgreen] (0,-1.5)  \hi 2 \lo 1 \hi 1 \lo 3 \lo 9 \hi 1;
      \fill[cred]   (0,-3.0)  \lo 2 \hi 1 \lo 1 \hi 3 \lo 9 \lo 1;
      \fill[cblue]  (0,-4.5)  \lo 2 \lo 1 \lo 1 \lo 3 \hi 9 \lo 1;
      \pattern[pattern=north east lines] (1, -1.5) rectangle +(3, 1);
      \pattern[pattern=north east lines] (4, -3.0) rectangle +(1, 1);
      \draw[cgreen] (0 + 0.5, -1) circle (3pt);
      \draw[cred]   (2 + 0.5, -2.5) circle (3pt);
      \draw[cblue]  (2 + 0.5, -4) circle (3pt);
    \end{tikzpicture}
  \end{center}
\end{frame} 

\begin{frame}{Héritage de priorité}
  Autre exemple:
  Soit 3 tâches et 2 ressources telle que A et B partage la première ressource et B et C partage la seconde:
  $$P(A) > P(B) > P(C)$$
  % CCBBCACBABC  
  \begin{center}
    \begin{tikzpicture}[scale=0.5]
      \timeline{11}{-5}{-1./A, -2.5/B, -4./C, }
      \fill[color=black!25] (6, -1.5) rectangle +(2, 1);
      \fill[color=black!25] (3, -3.0) rectangle +(4, 1);
      \fill[cgreen] (0,-1.5)  \lo 2 \lo 2 \lo 1 \hi 1 \lo 1 \lo 1 \hi 1 \lo 1 \lo 1;
      \fill[cred] (0,-3.0)  \lo 2 \hi 2 \lo 1 \lo 1 \lo 1 \hi 1 \lo 1 \hi 1 \lo 1;
      \fill[cblue] (0,-4.5)  \hi 2 \lo 2 \hi 1 \lo 1 \hi 1 \lo 1 \lo 1 \lo 1 \hi 1;
      \pattern[pattern=north east lines] (1, -4.5) rectangle +(6, 1);
      \pattern[pattern=north east lines] (7, -3.0) rectangle +(3, 1);
      \pattern[pattern=north west lines] (3, -3.0) rectangle +(5, 1);
      \pattern[pattern=north west lines] (8, -1.5) rectangle +(1, 1);
      \draw[cgreen] (5 + 0.5, -1) circle (3pt);
      \draw[cred]   (2 + 0.5, -2.5) circle (3pt);
      \draw[cblue]  (0 + 0.5, -4) circle (3pt);
    \end{tikzpicture}
  \end{center}
\end{frame} 

\begin{frame}{Héritage de priorité} 
  Concernant le temps de réponse  des tâches dans un système intégrant
  l'héritage de priorité:
  \begin{itemize}
  \item Le  temps de blocage $B_i$  d'une tâche de  haute priorité par
    une tâche de plus basse priorité nominale est borné
  \item ...  mais le calcul de  ce temps de blocage  maximum peut être
    très complexe
  \item Condition nécessaire d'ordonnançabilité par un algorithme Rate
    Monotonic :
    $$\forall i \in tasks \left( \sum_{j = 1..i} \frac{C_k}{T_k} \right) + \frac{B_i}{T_i} \le i(2^{1/i} - 1)$$
  \end{itemize}
\end{frame}

\begin{frame}{Problème rémanent} 
  \begin{itemize}
  \item Blocages en chaîne (cf notre exemple avec 3 tâches)
  \item Deadlock
  \end{itemize}
\note{Ajouter deux exemples?}
\end{frame}

\subsection{Priority ceilling}

\begin{frame}{Priorité plafonnée}
  \begin{itemize} 
  \item Aussi appellée \emph{Priority ceilling}
  \item Introduit  à la  fin des années  80 pour résoudre  le problème
    d'inversion de priorité tout en prévenant l'occurence de deadlocks
    et de blocages en chaîne
  \item Amélioration  du protocole d'héritage de priorité  : une tâche
    ne peut pas entrer dans une section critique s'il y a un sémaphore
    acquis qui pourrait la bloquer
  \item Principe : on attribue à chaque sémaphore une priorité plafond
    égale  à  la  plus   haute  priorité  des  tâches  qui  pourraient
    l'acquérir. Une  tâche ne pourra  entrer dans la  section critique
    que si  elle possède une  priorité supérieure à toutes  celles des
    priorités plafond des sémaphores acquis par les autres tâches
  \end{itemize} 
\end{frame} 

\begin{frame}{Algorithme}
  \begin{itemize} 
  \item  On attribue  à chaque  sémaphore $S_k$  une  priorité plafond
    $C(S_k)$ égale à la plus haute priorité des tâches susceptibles de
    l'acquérir
  \item Soit $T_i$  la tâche prête de plus  haute priorité : l'accès
    au processeur est donné à $T_i$
  \item soit $S*$ le sémaphore dont la priorité plafond $C(S*)$ est la
    plus grande parmi  tous les sémaphores déjà acquis  par des tâches
    autres que $T_i$
  \item Pour entrer dans une  section critique gardée par un sémaphore
    $S_k$, $T_i$ doit avoir une  priorité supérieure à $C(S*)$. Si $Pi
    \le C(S*)$, l'accès  à $S_k$ est interdit et on  dit que $T_i$ est
    bloquée sur $S*$ par la tâche qui possède $S*$
  \end{itemize} 
\end{frame} 

\begin{frame}{Exemple}
  \begin{center}
    \begin{tabular}{ccccc}
      \hline
      Tâche & Arrivée & Priorité & Capacité & Ressources \\
      \hline
      A & 5 & 1 & 4 & S0 après 1 et pendant 1, puis S1 après 3 et pendant 1\\
      B & 2 & 2 & 3 & S2 après 1 et pendant 1 \\
      C & 0 & 3 & 7 & S2 après 1 et pendant 5, et S1 après 2 et pendant 2 (de manière imbriquée)\\
      \hline
    \end{tabular}
  \end{center}
  \note{Est-ce que S0 est utile?}
  Les priorités plafond sont donc: 
  \begin{itemize}
  \item $C(S_0) = P_A$ 
  \item $C(S_1) = P_A$ 
  \item $C(S_2) = P_B$ 
  \end{itemize}
\end{frame} 

\begin{frame}{Sans mécanisme de protection}
%   %      AAAA
%   %   B        BB
%   % CC CC    CC  C
% 0 %       A
% 1 %    CC   A 
% 2 %  CCCCCCCCCCB 
\end{frame} 

\begin{frame}{Par héritage de priorité}
%   %      AAAA
%   %   B        BB
%   % CC CC    CC  C
% 0 %       A
% 1 %    CC   A 
% 2 %  CCCCCCCCCCB 
\end{frame} 

\begin{frame}{Par priorité plafonnée}
%   %      A AAA
%   %   B        BB
%   % CC CC C   C  C
% 0 %        A
% 1 %    CC    A 
% 2 %  CCCCCCCCCCB 
\end{frame} 

\begin{frame}{Priorité plafonnée} 
  \begin{itemize}
  \item à t = 0, T2 est activée et démarre.
  \item à t =  1, Le sémaphore S2 est demandé et  obtenu (il n'y a pas
    d'autre ressource en cours d'utilisation)
  \item à t = 2, T1 est activée et préempte T2
  \item à t = 3, T1 demande  S2 et est bloquée par le protocole car la
    priorité de T1 n'est pas supérieure à la priorité plafond $C(S2)=P1$
    (S2 est le seul sémaphore acquis à t2) T2 hérite de la priorité de
    T1 et redémarre
  \item à t  = 4, T2 demande  et obtient le sémaphore S1  , car aucune
    autre tâche ne détient de ressource
  \item à t = 5 T0 est réveillée et préempte T2
  \item  à t =  6, T0  demande le  sémaphore S0  qui n'est  détenu par
    aucune autre tâche. Le protocole  bloque néanmoins T0 parce que sa
    priorité n'est  pas supérieure à  la plus grande  priorité plafond
    des sémaphores  déjà détenus  (S2 et  S1) : P0.   T2 hérite  de la
    priorité de T0 et peut redémarrer
  \item à t = 7 , T2 relâche S1. Sa priorité p2 est ramenée à P1, plus
    haute priorité des tâches bloquées par T2. T0 peut alors préempter
    T2 et acquérir S0
  \item à t = 9 quand T0 demande S1, le seul sémaphore encore tenu est
    S2 avec  une priorité plafond  $P1 \leftarrow T0$ (priorité  $P0 >
    P1$) obtient S1
  \item à  t =  10, T0  se termine. T2  peut reprendre  son exécution,
    toujours avec la même priorité P1
  \item à t =  11, T2 relâche S2. Sa priorité est  ramenée à sa valeur
    nominale P2. T1 peut alors préempter T2 et redémarrer
  \item à t = 12, T1 se termine, T2 peut redémarrer et se terminer
  \end{itemize}
\end{frame}

\begin{frame}{Autre Exemple}
  Par héritage de priorité
  % CCBBACCAABAABC
  \begin{center}
    \begin{tikzpicture}[scale=0.5]
      \timeline{14}{-5}{-1./A, -2.5/B, -4./C, }
      \fill[cgreen]  (0,-1.5) \lo 2 \lo 2 \hi 1 \lo 2 \hi 2 \lo 1 \hi 2 \lo 1 \lo 1;
      \fill[cred]    (0,-3.0) \lo 2 \hi 2 \lo 1 \lo 2 \lo 2 \hi 1 \lo 2 \hi 1 \lo 1;
      \fill[cblue]   (0,-4.5) \hi 2 \lo 2 \lo 1 \hi 2 \lo 2 \lo 1 \lo 2 \lo 1 \hi 1;
      % Adapt to your needs:
      \foreach \i in {0}
        \draw[cgreen]  (\i + 0.5, -1.0) circle (3pt);
      \foreach \i in {2}
        \draw[cred]    (\i + 0.5, -2.5) circle (3pt);
      \foreach \i in {4}
        \draw[cblue]   (\i + 0.5, -4.0) circle (3pt);
      \pattern[pattern=north east lines] (1, -4.5) rectangle +(1, 1);
      \pattern[pattern=north east lines] (7, -1.5) rectangle +(1, 1);
      \pattern[pattern=north east lines] (4, -4.5) rectangle +(2, 1);
      \pattern[pattern=north west lines] (3, -3.0) rectangle +(1, 1);
      \pattern[pattern=north west lines] (9, -3.0) rectangle +(1, 1);
      \pattern[pattern=north west lines] (10, -1.5) rectangle +(1, 1);
    \end{tikzpicture}
  \end{center}
\end{frame} 

\begin{frame}{Autre Exemple}
  Par priorité plafonnée:
  % CCBCACAAAABBBC
  \begin{center}
    \begin{tikzpicture}[scale=0.5]
      \timeline{14}{-5}{-1./A, -2.5/B, -4./C, }
      \fill[cgreen]  (0,-1.5) \lo 2 \lo 1 \lo 1 \hi 1 \lo 1 \hi 4 \lo 3 \lo 1;
      \fill[cred]    (0,-3.0) \lo 2 \hi 1 \lo 1 \lo 1 \lo 1 \lo 4 \hi 3 \lo 1;
      \fill[cblue]   (0,-4.5) \hi 2 \lo 1 \hi 1 \lo 1 \hi 1 \lo 4 \lo 3 \hi 1;
      % Adapt to your needs:
      \foreach \i in {0}
        \draw[cgreen]  (\i + 0.5, -1.0) circle (3pt);
      \foreach \i in {2}
        \draw[cred]    (\i + 0.5, -2.5) circle (3pt);
      \foreach \i in {4}
        \draw[cblue]   (\i + 0.5, -4.0) circle (3pt);
      \pattern[pattern=north east lines] (1, -4.5) rectangle +(1, 1);
      \pattern[pattern=north east lines] (3, -4.5) rectangle +(1, 1);
      \pattern[pattern=north east lines] (5, -4.5) rectangle +(1, 1);
      \pattern[pattern=north east lines] (6, -1.5) rectangle +(1, 1);
      \pattern[pattern=north west lines] (8, -1.5) rectangle +(1, 1);
      \pattern[pattern=north west lines] (10, -3.0) rectangle +(2, 1);
    \end{tikzpicture}
  \end{center}
\end{frame} 

\begin{frame}{Priorité plafonnée} 
  On a  le même critère d'ordonnançabilité  par RM que dans  le cas du
  protocole d'héritage de priorité:
  $$\forall i \in tasks \left( \sum_{j = 1..i} \frac{C_k}{T_k} \right) + \frac{B_i}{T_i} \le i(2^{1/i} - 1)$$
  mais le  calcul du  temps de blocage  maximum de chaque  tâche est
  plus simple :
    \begin{itemize}
    \item on  peut démontrer  que le temps  de bloquage  maximum $B_i$
      d'une tâche  $T_i$ est la durée  de la plus  longue des sections
      critiques ($\max D_{j,k}$) parmi celles appartenant à des tâches
      de  priorité inférieure à  $P_i$ et  gardées par  des sémaphores
      dont la priorité plafond est supérieure ou égale à $P_i$
      $$B_i = \max D_{j,k} | P_j < P_i, C(S_k) \ge P_i$$ 
    \end{itemize}
\end{frame}

\section{Autres mécanismes de gestion d'accès concurrents}

\subsection{Désactivation de l'ordonnanceur}

\note{Pour chacun des mécanisme, donner les fonction dasn plusieurs API, faire un exemple ou mieux, donner le code: Posix, Java, Xenomai, ucosII, vxWorks}

\subsection{Sémaphore}

\begin{frame}{Sémaphore}
  \begin{itemize} 
  \item  Différence entre un  mutex et  un semaphore  binaire: presque
    aucune.
  \item Parfois le sémaphore  binaire est utilisé pour implémentéer le
    mutex.
  \item Toutefois,  d'un point de  vue sémantique, on pourrait  que le
    mutex  permet  d'avoir un  morceau  de  code mutuelement  exclusif
    tandis que  le sémaphore est  une section de  code limité à  uen 1
    ressource.
  \item  Notons  aussi  que  les  algorithme  d'héritage  de  priorité
    néxistent pas sur les sémaphores
  \end{itemize} 
\end{frame} 

\subsection{Mutex réentrant}

\begin{frame}{Mutex Réentrant}
  \begin{itemize} 
  \item Idem  mutex, mais  si la même  tâche tente de  revérouiller le
    même mutex, le mutex est non-bloquant.
  \item Dans le cas  d'un mutex non-réentrant, ceci entraine forcement
    un dead-lock.
  \item  Un sémaphore est maintenu pour connaitre le
    nombre de passage.
  \end{itemize} 
\end{frame} 

\subsection{R/W Lock}

\begin{frame}[fragile]{Read/Write Lock}
\note{\url{http://en.wikipedia.org/wiki/Readers-writers\_problem}}

Permet de limiter le phénomène de latence en dimiminuant le nombre de sections critiques.

Solution 1 (\emph{reader preference}):
\begin{columns}
  \begin{column} {5cm}
    \begin{lstlisting} 
void read_lock() {
  // mutex protege read_count
  lock(mutex);
  readcount++;
  if (readcount == 1)
    lock(w);
  unlock(mutex);
}
    \end{lstlisting} 
  \end{column}
  \begin{column} {5cm}
    \begin{lstlisting} 
void read_unlock() {
  lock(mutex);
  readcount--;
  if (readcount = 0)
    unlock(w);
  unlock(mutex);
}
void write_lock() {
   lock(w);
}
void write_unlock() {
  unlock(w);
}
    \end{lstlisting} 
  \end{column}
\end{columns}
\end{frame} 

\begin{frame}[fragile]{Read/Write Lock}
  Problème: un accès en écriture doit attendre que toutes les lectures
  soient terminées. Solution 2 (\emph{writer preference}):
  \begin{columns}
    \begin{column} {5cm}
      \begin{lstlisting} 
void read_lock() {
  lock(r);
  lock(mutex);
  readcount++;
  if (readcount == 1)
     lock(w);
  unlock(mutex);
  unlock(r);
  // r n'est pas bloque durant la lecture
}
       \end{lstlisting} 
     \end{column}
     \begin{column} {5cm}
       \begin{lstlisting} 
void read_unlock() {
  lock(mutex);
  readcount--;
  if (readcount == 0)
     unlock(w);
  unlock(mutex);
}
void write_lock() {
   lock(r);
   lock(w);
}
void write_unlock() {
  unlock(w);
  unlock(r);
}
      \end{lstlisting} 
    \end{column}
  \end{columns}
\end{frame} 

\subsection{Rendez-vous ou barrier}
\begin{frame}[fragile]{Rendez-vous}
Permet de synchroniser deux tâches. La première tâche arrivée à la barrière attend la seconde.
\begin{lstlisting} 
void init() {
  lock(m1);
  lock(m2);
}
\end{lstlisting}
Tâche 1:
\begin{lstlisting} 
  unlock(m1);
  lock(m2);
\end{lstlisting} 
Tâche 2:
\begin{lstlisting} 
  unlock(m2);
  lock(m1);
\end{lstlisting} 
\end{frame}

\subsection{Condition}

\begin{frame}[fragile]{Conditions}
Peut être considéré comme des \emph{rendez-vous} à sens unique. Si une tâche attend, elle est débloquée, sinon, aucun effet. Très utilisée pour le pattern des \cmd{work-thread}
\begin{lstlisting} 
void init() {
  lock(m);
}

void wait() {
  lock(m);
}

void signal() {
  unlock(m);
  try_lock(m);
}

// broadcast debloque tous les waiters que signal en deloque un uniquement
void broadcast()  {
  // Plus complexe, il faut un mutex par waiters. 
}
\end{lstlisting} 
\end{frame} 


\subsection{Buffer circulaire}
\begin{frame}[fragile]{Buffer circulaire}
\begin{lstlisting} 
init
  w = r = 0;

write(c);
  if ((w + 1) % size == r )
    ; //buffer is full
  buf[w] = c;
  w = (w + 1) % size;

read():
  if (w == r)
    ; //buffer is empty
  ret = buf[r];
  r = (r + 1) % size;
\end{lstlisting} 

Pas de mutex: peut-être utilisé dans une interruption
\end{frame} 

\subsection{Queue}
\begin{frame}{Queue}
Idem  Buffer circulaire,  mais avec  un tableau  de structure.  Il est
aussi possible de faire des Queue d'objets de tailles différente. Dans
ce cas,  faire très attention à l'allocation  des objets. L'allocation
dynamique est rarement une opération très bornée dans le temps.
\end{frame} 

\subsection{Spin Lock}
\begin{frame}[fragile]{Spin Lock}
Attente active
\begin{lstlisting} 
lock(m)
  while(atomic_test_and_set(m))
     ;

unlock(m)
  m = 0;
\end{lstlisting} 

Uniquement dans le  cas d'architecture multicoeur. Cas ou  le spin est
locké par un autre processeur pour  une courte période. Dans ce cas le
réordonnacement à un cout plus important que l'attente.

Dans le cas  ou un spin lock  serait bloqué par une tâche  sur le même
processeur, il  faut théoriquement attendre la préemption  pour que la
tâche soit reschedulée.  Si la tâche est de  haute priorité, cela peut
conduire à un  dead lock. (Il y a très souvent  un garde-fou contre ce
problème).
\end{frame} 

\subsection{Algorithmes non-bloquants}
\begin{frame}{Algorithmes non-bloquants}
Algorithme thread-safe n'utilisant pas de sections ciritques. 
Ces algorithmes utilisent souvant des instructions atomiques proposés par certains processeurs
Par conséquant, ils sont peu portables
Souvent utilisé dans les base de données
\end{frame} 

\subsection{RCU}
\begin{frame}[fragile]{Read-Copy-Update}
Type d'algorithme non bloquant
Particulièrement utile pour les manipulation de listes:


\begin{lstlisting} 

struct rcu_t {
   struct a_t a;
   int count_usage = 0;
   bool obsolete = false;
}
rcu_t *a = ammloc(sizeof(rcu_t)); 

struct a_t *a = malloc(sizeof(struct a_t));
int *count_usage = malloc(sizeof(int));
*count_usage = 0;
mutex m;

read_a() {
   readlock() {
     rcu_t *ptr = a;
     ptr->count_usage++;
     return ptr;
   }
   Do_something with ptr;

   read_unlock(ptr) {
     ptr->count_usage--;
     if (ptr->obsolete && ! ptr->count_usage)
        free(ptr);
    }
}

write_a() { 
   struct rcu_t *a3 = a;
   struct rcu_t *a2 = malloc(sizeof(struct rcu_t));
   memcpy(a2, a);
   modify a2;   
   a = a2;  
   a3.obsolete = true;
   if (!a3.count_usage)
      free(ptr);
}
\end{lstlisting} 
\end{frame} 
